\سؤال{\lr{Decorator}}

برای نوشتن این کلاس‌ها، همان‌طور که در ویدیوی آموزشی گفته شد، یک مجموعه تست به ما داده شده است و باید قدم به قدم و مرحله به مرحله شروع به برطرف کردن خطاها کنیم تا در نهایت همه‌ی تست‌ها پاس شوند.

باید یک \lr{interface} به نام \lr{Beverage} ایجاد کنیم که یک فیلد به نام \lr{description} دارد و  سه متد \lr{getDescription}، \lr{cost} را پیاده‌سازی کرده است. 

\begin{Verbatim}[tabsize=4]
public interface Beverage {
	public String description = null;
	public String getDescription();
	public Double cost();
}
\end{Verbatim}

حال باید چهار کلاس \lr{HouseBlend}، \lr{DarkRoast}، \lr{Espresso} و \lr{Decaf} را ایجاد نماییم و جزئیات (شامل اسم و قیمت با توجه به تست‌های داده‌شده) را اضافه کنیم. برای مثال کلاس \lr{Espresso} به شکل زیر است:

\begin{Verbatim}[tabsize=4]
public class Espresso implements Beverage {
	@Override
	public String getDescription() {
		return "Delicious Espresso";
	}
	
	@Override
	public Double cost() {
		return 1.99 ;
	}
}
\end{Verbatim}

در این مرحله باید یک کلاس \lr{decorator} به نام \lr{CondimentDecorator} را اضافه کنیم تا با ارتباطی که با \lr{Beverage} دارد، قابلیت انعطاف و پویایی را اضافه کند. 



\begin{Verbatim}[tabsize=4]
public abstract class CondimentDecorator  implements Beverage {
	private Beverage beverage;
	
	public CondimentDecorator(Beverage beverage) {
		this.beverage = beverage;
	}
	
	@Override
	public String getDescription() {
		return beverage.getDescription();
	}
	
	@Override
	public Double cost() {
		return beverage.cost();
	}
}
\end{Verbatim}

حال با استفاده از \lr{super} در زیر کلاس‌های مربوط به \lr{decorator} این پویایی به برنامه اضافه می‌شود.

برای مثال، کلاس \lr{Mocha} به شکل زیر است:

\begin{Verbatim}[tabsize=4]
public class Mocha extends CondimentDecorator {

	public Mocha(Beverage beverage) {
		super(beverage);
	}
	
	public String getDescription() {
		return super.getDescription() + " with mocha";
	}
	
	public Double cost() {
		return super.cost() + 0.20 ;
	}
}
\end{Verbatim}
در نهایت تمامی تست‌ها پاس شد.

\newpage
\سؤال{\lr{Bridge}}

در این قسمت، برای این‌که کارها و مسئولیت‌ها را به کلاس‌های متفاوت (\lr{Abstract} و پیاده‌سازی‌ها) را از یک‌دیگر جدا کنیم، از این الگوی طراحی استفاده می‌کنیم.

با توجه به \lr{Test Driven Development} ابتدا باید تست‌ها را طراحی کنیم و سپس اقدام به نوشتن منطق برنامه کنیم.

\begin{Verbatim}[tabsize=4]
public class PowerTest {
	@Test
	public void testPowerStandardRecursive() {
		Power power = new PowerImplStandard(new MultImplRecursive());
		Assert.assertEquals(81, power.power(3, 4), 0) ;
	}
	
	@Test
	public void testPowerRecursiveRecursive() {
		Power power = new PowerImplRecursive(new MultImplRecursive());
		Assert.assertEquals(81, power.power(3, 4), 0) ;
	}
	
	@Test
	public void testPowerRecursiveStandard(){
		Power power = new PowerImplRecursive(new MultImplStandard());
		Assert.assertEquals(81, power.power(3, 4), 0) ;
	}
	
	@Test
	public void testPowerStandardStandard(){
		Power power = new PowerImplStandard(new MultImplStandard());
		Assert.assertEquals(81, power.power(3, 4), 0) ;
	}
}

\end{Verbatim}

کلاس مربوط به توان را \lr{Abstract} در نظر می‌گیریم زیرا باید یک فیلد از \lr{Interface} ضرب داشته باشد و در جایی که نیاز داریم، از پیاده‌سازی‌ها مربوط به آن استفاده کند:

\begin{Verbatim}[tabsize=4]
public abstract class Power {
	protected Mult mult;
	
	public Power(Mult mult) {
		this.mult = mult;
	}
	
	public abstract Integer power(int a, int b);
}

\end{Verbatim}

در قسمت بعدی \lr{interface} ضرب را پیاده‌سازی می‌کنیم:

\begin{Verbatim}[tabsize=4]
public interface Mult {
	public Integer mult(int a, int b);
}
\end{Verbatim}

حال کافی است پیاده‌سازی‌های مختلف ضرب و توان را به کار ببندیم.

پیاده‌سازی‌های انجام شده:
\begin{itemize}
	\item پیاده‌سازی معمولی
	\item پیاده‌سازی بازگشتی
\end{itemize}

\textbf{پیاده‌سازی بازگشتی ضرب}
\begin{Verbatim}[tabsize=4]
public class MultImplRecursive implements Mult {
	@Override
	public Integer mult(int a, int b) {
		if (b == 0) {
			return 0;
		}
		return a + mult(a, b - 1);
	}
}
\end{Verbatim}

\textbf{پیاده‌سازی معمولی ضرب}
\begin{Verbatim}[tabsize=4]
public class MultImplStandard implements Mult {
	
	@Override
	public Integer mult(int a, int b) {
		int result = 0;
		
		while(b != 0) {
			result += a;
			b -= 1;
		}
		
		return result;
	}
}
\end{Verbatim}


\textbf{پیاده‌سازی بازگشتی توان}
\begin{Verbatim}[tabsize=4]
public class PowerImplRecursive extends Power {

	public PowerImplRecursive(Mult mult) {
		super(mult);
	}
	
	@Override
	public Integer power(int base, int exponent) {
		if (exponent != 0) {
			return mult.mult(base, power(base, exponent - 1));
		}
		else {
			return 1;
		}
	}
}
\end{Verbatim}

\textbf{پیاده‌سازی معمولی توان}:
\begin{Verbatim}[tabsize=4]
public class PowerImplStandard extends Power {
	
	public PowerImplStandard(Mult mult) {
		super(mult);
	}
	
	@Override
	public Integer power(int base, int exponent) {
		int result = 1;
		while (exponent != 0) {
			result = mult.mult(result, base);
			--exponent;
		}
		return result;
	}
}
\end{Verbatim}

به این ترتیب تست‌های مربوط به این قسمت نیز پاس می‌شوند.